\documentclass[journal,letterpaper]{IEEEtran}
\usepackage[letterpaper, left=0.65in, right=0.65in, bottom=0.7in, top=0.7in]{geometry}
\usepackage{stix}
\usepackage{siunitx}
\usepackage{tabularray}
\usepackage{amsmath}
\usepackage{graphicx}
\usepackage{float}
\usepackage{fancyhdr}
\usepackage[none]{hyphenat}
\usepackage[hidelinks]{hyperref}
\usepackage{import}
\usepackage{transparent}
%\usepackage{microtype}

\graphicspath{ {./figures/} }

\newcommand{\incfig}[1]{%
    \centering
    \def\svgwidth{3.5in}
    \import{./figures/}{#1.pdf_tex}
}

\sisetup{per-mode = symbol, inter-unit-product = \ensuremath{ { } \cdot { } }}

\bibliographystyle{IEEEtran}

\pagestyle{fancy}
\fancyhf{}
\renewcommand{\headrulewidth}{0pt}
\rhead{\thepage}
\lhead{Section 11832 Lab 1}

\setlength{\columnsep}{0.2in}
\setlength{\columnwidth}{3.5in}

\begin{document}
\title{Characterizing the Viscous Loss of a Wind Tunnel}

\author{\IEEEauthorblockN{\huge{Last Name, First Name \\}}
\IEEEauthorblockA{
Section 11832 \quad September 19, 2023}
}

\maketitle
\thispagestyle{empty}

\begin{abstract}
Abstract
\end{abstract}

\begin{IEEEkeywords}
calibration, flow velocity, viscous loss, wind tunnel
\end{IEEEkeywords}


\section{Introduction}


\IEEEPARstart{W}{ind} tunnels have proven to be a valuable tool for predicting the aerodynamic capabilities of objects for over a century \cite{lecture}.
Air is moved through the tunnel and the interactions between the airflow and the object can be used to simulate the actual flight characteristics of the object.
As the velocity of the flowing air can be a major influence on the performance of the object in the wind tunnel, it is vital to have an accurate understanding of the velocity of the air flow.
This can be complicated by the presence of viscous losses and the flow conditioning screens which affect the kinetic energy of the fluid.
The use of a pitot tube to measure the air velocity would be the simplest choice, but the physical presence of a model in the wind tunnel would disrupt the flow to the pitot tube and cause inaccurate readings.
This lab determined a tunnel calibration coefficient, $K$, that accounts for the viscous losses and heat transfer losses in the wind tunnel.
This was accomplished by measuring the pressure gradient across multiple fan speeds and using the first law of thermodynamics to establish a relationship between the dynamic pressure and the static pressure accounting for losses.
With the correction factor, measurements of the static pressure at the entrance of the test section can be used to determine the flow velocity.


\section{Procedure}


Procedure


\section{Discussion}


Discussion


\section{Conclusion}


Conclusion


\section*{Appendix A: Uncertainty Calculation}


Uncertainty


\begin{thebibliography}{9}
    \bibitem{b1} G. O. Young, ``Synthetic structure of industrial plastics (Book style with paper title and editor),'' in \textit{Plastics}, 2nd ed. vol. 3, J. Peters, Ed. New York: McGraw-Hill, 1964, pp. 15--64.
    \bibitem{b2} W.-K. Chen, \textit{Linear Networks and Systems} (Book style).	Belmont, CA: Wadsworth, 1993, pp. 123--135.
    \bibitem{b3} H. Poor, \textit{An Introduction to Signal Detection and Estimation}. New York: Springer-Verlag, 1985, ch. 4.
    \bibitem{b4} B. Smith, ``An approach to graphs of linear forms (Unpublished work style),'' unpublished.
    \bibitem{b5} E. H. Miller, ``A note on reflector arrays (Periodical style—Accepted for publication),'' \textit{IEEE Trans. Antennas Propagat.}, to be published.
\end{thebibliography}

\end{document}